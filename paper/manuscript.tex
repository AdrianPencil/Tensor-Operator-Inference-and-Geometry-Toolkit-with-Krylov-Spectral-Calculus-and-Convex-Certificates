\documentclass[11pt]{article}

\usepackage[a4paper,margin=1.15in]{geometry}
\usepackage{setspace}
\usepackage{microtype}
\usepackage{amsmath,amssymb,mathtools,bm}
\usepackage{physics}
\usepackage{hyperref}

\setstretch{1.12}

% ----------------------------
% Math macros (minimal)
% ----------------------------
\newcommand{\RR}{\mathbb{R}}
\newcommand{\CC}{\mathbb{C}}
\newcommand{\Id}{\mathbb{I}}
\newcommand{\T}{^\mathsf{T}}
\newcommand{\adj}{^{\ast}}
\newcommand{\ip}[2]{\left\langle #1,\,#2\right\rangle}
\newcommand{\normF}[1]{\left\lVert #1 \right\rVert_{\mathrm{F}}}
\newcommand{\vecop}{\mathrm{vec}}
\newcommand{\kron}{\otimes}

\title{\textbf{Tensor Inference Geometry: Math-First Operator, Manifold, and Tensor-Network Tooling}}
\author{tensor-inference-geometry (tig)}
\date{\today}

\begin{document}
\maketitle

\begin{abstract}
We present \emph{tensor-inference-geometry} (tig), a math-first library for tensor inference and geometry-aware optimization.
The project is organized around operator-theoretic contracts (matvec and adjoint pairing), Fr\'echet-derivative views (JVP/VJP), matrix-free Krylov methods, matrix manifolds (Stiefel, Grassmann, fixed-rank), and tensor-network representations (MPS/MPO).
Implementation is TensorFlow-first with float64, while preserving theoretical structure rather than default flattening.
\end{abstract}

\section{Scope and organizing principles}
tig is designed as a theory-aligned codebase: the primary objects are multilinear maps, linear operators with adjoints, and structured tensor representations.

\paragraph{Non-negotiable rules.}
\begin{itemize}
  \item \textbf{Math-first definitions:} every implementation is backed by an explicit operator identity.
  \item \textbf{Adjoint pairing contract:} for operators $A$, enforce $\ip{Ax}{y} = \ip{x}{A\adj y}$.
  \item \textbf{Matrix-free by default:} dense constructions are diagnostics only.
  \item \textbf{Tensor structure is preserved:} use contraction algebra and structured formats when possible.
\end{itemize}

\section{Tensor spaces and multilinear maps}
For spaces $V_1,\dots,V_k$, tensor products $V_1\kron\dots\kron V_k$ represent multilinear maps via the universal property.
Coordinates $T_{a_1\dots a_k}$ are implemented as multi-index arrays and contractions are realized using Einstein summation.

\section{Fr\'echet derivatives and sensitivity operators}
For $F:X\to Y$, Fr\'echet differentiability at $x$ provides a bounded linear map $DF(x):X\to Y$ satisfying
\begin{equation}
  F(x+h) = F(x) + DF(x)[h] + o(\norm{h}).
\end{equation}
We expose two matrix-free primitives:
\begin{align}
  \text{JVP: } & v \mapsto DF(x)[v], \\
  \text{VJP: } & u \mapsto DF(x)\adj[u],
\end{align}
together with the adjoint identity
\begin{equation}
  \ip{DF(x)[v]}{u}_Y = \ip{v}{DF(x)\adj[u]}_X.
\end{equation}

\section{Operator theory and Krylov solvers}
A linear operator $A$ is represented by $(\mathrm{matvec}, \mathrm{rmatvec})$, enabling Krylov methods without forming matrices.
For SPD systems, conjugate gradients (CG) applies to $Ax=b$ with convergence governed by conditioning.

\section{Inverse problems and regularization}
Given data $y$ and model $y \approx F(x)$, MAP-type objectives take the form
\begin{equation}
  \hat{x} = \arg\min_x \frac{1}{2\sigma^2}\norm{F(x)-y}^2 + \lambda R(x).
\end{equation}
tig includes proximal operators for standard regularizers and minimal ADMM/prox toolchains for convex structure.

\section{Matrix manifolds and geometry-aware optimization}
We implement retractions and tangent-space operations on matrix manifolds:
\begin{align}
  \mathrm{St}(n,p) &= \{X\in\RR^{n\times p}: X\T X = \Id_p\},\\
  \mathrm{Gr}(n,p) &= \mathrm{St}(n,p)/\mathrm{O}(p).
\end{align}
Optimization uses Riemannian gradients and retractions (for example QR-based).

\section{Tensor networks}
Tensor networks provide structured representations avoiding exponential blowup:
\begin{itemize}
  \item MPS for vectors in $(\RR^d)^{\kron L}$
  \item MPO for operators in tensor-product spaces
\end{itemize}
Core checks compare contractions to dense reconstructions on small instances.

\section{Stochastic processes, filtering, and spectral framing}
SDE tooling centers around It\^{o} increments and Euler--Maruyama discretizations, with baseline linear-Gaussian filtering.
Spectral modules expose expmv, resolvent operators, and trace/logdet proxies to support operator calculus.

\section{Compute contracts and reproducibility}
The code treats compute constraints as mathematical invariants:
vectorization identities, layout descriptors, and FLOP/byte proxies accompany correctness checks.
All stochastic components use explicit RNG objects for deterministic replay.

\section{Conclusion}
tig is a coherent spine connecting tensor algebra, operator theory, inverse problems, geometry, and structured computation.
The repository is designed to remain math-first while enabling practical experiments through TensorFlow-backed tensor operations.

\bibliographystyle{plain}
\bibliography{references}

\end{document}
